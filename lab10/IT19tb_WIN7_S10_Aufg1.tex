\documentclass{article}

\usepackage[utf8]{inputenc}

% to set spacing of page margins
\usepackage[margin=2cm,nohead]{geometry}
% to include images
\usepackage{graphicx}
% to automatically create paragraphs when doing so in the code
\usepackage{parskip}
% additional math stuff
\usepackage{amsmath}
% line spacing
\usepackage{setspace}
% for checkmarks
\usepackage{amssymb}

% to not indent new paragraphs
\setlength{\parindent}{0pt}

%\doublespacing
\onehalfspacing

\begin{document}

\section*{IT19tb WIN7 S10 Aufgabe 1}
Leo Rudin \& Stefan Teodoropol

Ax = b mit

\[
A =
\begin{bmatrix} 
8 & 5 & 2 \\
5 & 9 & 1 \\
4 & 2 & 7 \\
\end{bmatrix}
b =
\begin{bmatrix} 
19 \\
5 \\
34 \\
\end{bmatrix}
\]

Vektornorm: \(||x||_\infty = \text{max}(|x_1|,...,|x_n|)\)\\
Matrixnorm: \(||x||_\infty = \text{max}(|x_{11}|+...+|x_{1n}|,...,|x_{n1}|+...+|x_{nn}|)\)

\subsection*{a)}

Überprüfung mittels der Zeilen:\\
\(8 > 5 + 2 \checkmark\)\\
\(9 > 1 + 5 \checkmark\)\\
\(7 > 4 + 2 \checkmark\)

Diese Bedingung ist erfüllt, somit sollte das System mittels dem Jacobi-Verfahren konvergieren.

\newpage
\subsection*{b)}

Gleichung Umstellen nach Komponenten:\\
\(8x_1 + 5x_2 + 2x_3 = 19 \rightarrow x_1 = \frac{19}{8} - \frac{5}{8}x_2 - \frac{2}{8}x_3\)\\
\(5x_1 + 9x_2 + 1x_3 = 5 \rightarrow x_2 = \frac{5}{9} - \frac{5}{9}x_1 - \frac{1}{9}x_3\)\\
\(4x_1 + 2x_2 + 7x_3 = 34 \rightarrow x_3 = \frac{34}{7} - \frac{4}{7}x_1 - \frac{2}{7}x_2\)

\vspace{5mm}

Aufstellung der Fixpunktgleichung:

\(
\begin{bmatrix} 
x_1\\
x_2\\
x_3\\
\end{bmatrix}
=
\begin{bmatrix} 
0 & -\frac{5}{8} & -\frac{2}{8} \\
-\frac{5}{9} & 0 & -\frac{1}{9} \\
-\frac{4}{7} & -\frac{2}{7} & 0 \\
\end{bmatrix}
*
\begin{bmatrix} 
x_1\\
x_2\\
x_3\\
\end{bmatrix}
+
\begin{bmatrix} 
\frac{19}{8}\\
\frac{5}{9}\\
\frac{34}{7}\\
\end{bmatrix}
\)

\vspace{5mm}

Durchführung der Iteration:

\(
x^{(0)} = 
\begin{bmatrix} 
1\\
-1\\
3\\
\end{bmatrix}
\)

\(
x^{(1)} = 
\begin{bmatrix} 
0 & -\frac{5}{8} & -\frac{2}{8} \\
-\frac{5}{9} & 0 & -\frac{1}{9} \\
-\frac{4}{7} & -\frac{2}{7} & 0 \\
\end{bmatrix}
*
\begin{bmatrix} 
1\\
-1\\
3\\
\end{bmatrix}
+
\begin{bmatrix} 
\frac{19}{8}\\
\frac{5}{9}\\
\frac{34}{7}\\
\end{bmatrix}
=
\begin{bmatrix} 
2.25\\
-\frac{1}{3}\\
4.57\\
\end{bmatrix}
\)

\(
x^{(2)} = 
\begin{bmatrix} 
0 & -\frac{5}{8} & -\frac{2}{8} \\
-\frac{5}{9} & 0 & -\frac{1}{9} \\
-\frac{4}{7} & -\frac{2}{7} & 0 \\
\end{bmatrix}
*
\begin{bmatrix} 
2.25\\
-\frac{1}{3}\\
4.57\\
\end{bmatrix}
+
\begin{bmatrix} 
\frac{19}{8}\\
\frac{5}{9}\\
\frac{34}{7}\\
\end{bmatrix}
=
\begin{bmatrix} 
1.44\\
-1.2\\
3.67\\
\end{bmatrix}
\)

\(
x^{(3)} = 
\begin{bmatrix} 
0 & -\frac{5}{8} & -\frac{2}{8} \\
-\frac{5}{9} & 0 & -\frac{1}{9} \\
-\frac{4}{7} & -\frac{2}{7} & 0 \\
\end{bmatrix}
*
\begin{bmatrix} 
1.44\\
-1.2\\
3.67\\
\end{bmatrix}
+
\begin{bmatrix} 
\frac{19}{8}\\
\frac{5}{9}\\
\frac{34}{7}\\
\end{bmatrix}
=
\begin{bmatrix} 
2.21\\
-0.65\\
4.38\\
\end{bmatrix}
\)

\newpage
\subsection*{c)}

Überprüfung der Kontraktionsbedingung:

\(||B||_\infty < 1\)\\
\(
||B||_\infty
=
||
\begin{bmatrix} 
0 & -\frac{5}{8} & -\frac{2}{8} \\
-\frac{5}{9} & 0 & -\frac{1}{9} \\
-\frac{4}{7} & -\frac{2}{7} & 0 \\
\end{bmatrix}
||_\infty
=
\text{max}\{\frac{7}{8},\frac{6}{9},\frac{6}{7}\}
=
\frac{7}{8}
\)

Da \(\frac{7}{8} < 1\) erfüllt ist, ist die Bedingung erfüllt.

\vspace{5mm}

A-Posteriori Abschätzung:

\(||x^{(n)} - \overset{\_}{x}||_\infty \leq \frac{||B||_\infty}{1 - ||B||_\infty} * ||x^{(n)} - x^{(n-1)}||_\infty\)

\(
||x^{(n)} - x^{(n-1)}||_\infty
=
||
\begin{bmatrix} 
2.21\\
-0.65\\
4.38\\
\end{bmatrix}
-
\begin{bmatrix} 
1.44\\
-1.2\\
3.67\\
\end{bmatrix}
||_\infty
=
\begin{bmatrix} 
0.77\\
0.55\\
0.71\\
\end{bmatrix}
=
\text{max}\{0.77,0.55,0.71\}
=
0.77
\)

\(\frac{\frac{7}{8}}{1 - \frac{7}{8}} * 0.77 = 5.39\)

Der maximale absolute Fehler beträgt 5.39.

\subsection*{d)}

A-Priori Abschätzung:

\(||x^{(n)} - \overset{\_}{x}||_\infty \leq \frac{||B||_\infty^n}{1 - ||B||_\infty} * ||x^{(1)} - x^{(0)}||_\infty\)

\(
||x^{(1)} - x^{(0)}||_\infty
=
||
\begin{bmatrix} 
2.25\\
-\frac{1}{3}\\
4.57\\
\end{bmatrix}
-
\begin{bmatrix} 
1\\
-1\\
3\\
\end{bmatrix}
||_\infty
=
||
\begin{bmatrix} 
1.25\\
\frac{2}{3}\\
1.57\\
\end{bmatrix}
||_\infty
=
\text{max}\{1.25,\frac{2}{3},1.57\} = 1.57
\)

\(10^{-4} \leq \frac{(\frac{7}{8})^n}{1 - \frac{7}{8}} * 1.57\)

\(\frac{10^{-4}}{8} \leq (\frac{7}{8})^n * 1.57\)

\(\frac{\frac{10^{-4}}{8}}{1.57} \leq (\frac{7}{8})^n\)

\(log_{\frac{7}{8}}\left(\frac{\frac{10^{-4}}{8}}{1.57}\right) \leq n\)

\(87.92 \leq n\)

Es braucht mindestens 88 Iterationen.

\newpage
\subsection*{e)}

\(
||x^{(3)} - x^{(2)}||_\infty
=
||
\begin{bmatrix} 
2.21\\
-0.65\\
4.38\\
\end{bmatrix}
-
\begin{bmatrix} 
1.44\\
-1.2\\
3.67\\
\end{bmatrix}
||_\infty
=
||
\begin{bmatrix} 
0.77\\
0.55\\
0.71\\
\end{bmatrix}
||_\infty
=
\text{max}\{0.77,0.55,0.71\} = 0.77
\)

\(10^{-4} \leq \frac{(\frac{7}{8})^n}{1 - \frac{7}{8}} * 0.77\)

\(\frac{\frac{10^{-4}}{8}}{0.77} \leq (\frac{7}{8})^n\)

\(log_{\frac{7}{8}}\left(\frac{\frac{10^{-4}}{8}}{0.77}\right) \leq n\)

\(82.59 \leq n\)

Es braucht mindestens 83 Iterationen.

\end{document}