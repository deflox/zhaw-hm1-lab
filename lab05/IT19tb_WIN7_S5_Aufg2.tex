\documentclass{article}
\usepackage[utf8]{inputenc}

\usepackage[utf8]{inputenc}
% to set spacing of page margins
\usepackage[margin=2cm,nohead]{geometry}
% to include images
\usepackage{graphicx}
% to automatically create paragraphs when doing so in the code
\usepackage{parskip}

% to not indent new paragraphs
\setlength{\parindent}{0pt}

% line spacing
\usepackage{setspace}
\doublespacing
%\onehalfspacing

\begin{document}

\section*{IT19tb WIN7 S5 Aufgabe 2}
Leo Rudin \& Stefan Teodoropol

Formel für Kugelvolumen: \(V_k = \frac{4}{3}*\pi*r^3\)\\
Formel für Kugelsegment: \(V_s = \frac{1}{3}*\pi*h^2*(3r-h)\)

Wir ziehen nun vom gesamten Kugelvolumen das Volumen des Wassers ab:\\
Kugelvolumen: \(V_k = \frac{4}{3}*\pi*(5m)^3 \approx 523.6m^3\)\\
Wasservolumen maximal: \(472m^3\)\\
Volumen des Kugelsegmentes: \(V_k - 472m^3 \approx 52.6m^3\)

Nun stellen wir folgende Gleichung mithilfe der Formel für das Kugelsegment auf:\\
\(\frac{1}{3}*\pi*h^2*(3r-h) = V_k - 472m^3\)\\
und stellen diese dann nach 0 um:\\
\(\frac{1}{3} \pi h^2*(3r-h) - (V_k - 472m^3) = 0\)

Wir leiten diese nun nach h ab:\\
\(f(h) = \frac{1}{3} \pi h^2*(3r-h) - (V_k - 472m^3)\)\\
\(f(h) = - \frac{1}{3} \pi h^3 + \pi r h^2 - (V_k - 472m^3)\)\\
\(f'(h) = - \pi h^2  + 2 \pi r h\)

Nun führen wir das Newtonverfahren durch:\\
\(h_{n+1} = h_n - \frac{f(h_n)}{f'(h_n)}\)\\
Startwert \(h_0\) = Höhe Kugel - Gegebener Startwert = \(1m\)

\(h_1 = 1m - \frac{\frac{1}{3} \pi (1m)^2*(15m-1m) - (\frac{4}{3}*\pi*(5m)^3 - 472m^3)}{- \pi (1m)^2  + 2 \pi * 5m * 1m} = 2.3064m\)\\
\(h_2 = 2.3064m - \frac{\frac{1}{3} \pi (2.3064m)^2*(15m-2.3064m) - (\frac{4}{3}*\pi*(5m)^3 - 472m^3)}{- \pi (2.3064m)^2  + 2 \pi * 5m * 2.3064m} = 1.9636m\)\\
\(h_3 = 1.9636m - \frac{\frac{1}{3} \pi (1.9636m)^2*(15m-1.9636m) - (\frac{4}{3}*\pi*(5m)^3 - 472m^3)}{- \pi (1.9636m)^2  + 2 \pi * 5m * 1.9636m} = 1.9427m\)\\
\(h_4 = 1.9427m - \frac{\frac{1}{3} \pi (1.9427m)^2*(15m-1.9427m) - (\frac{4}{3}*\pi*(5m)^3 - 472m^3)}{- \pi (1.9427m)^2  + 2 \pi * 5m * 1.9427m} = 1.9426m\)

Die Wasserhöhe darf höchstens 10m-1.9426m=8.0574m betragen.\\
Auf drei Kommastellen gerundet wäre das: 8.057m.

\end{document}
