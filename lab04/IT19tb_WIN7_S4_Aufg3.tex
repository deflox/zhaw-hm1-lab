\documentclass{article}

\usepackage[utf8]{inputenc}
\usepackage[margin=2cm,nohead]{geometry}
\usepackage{graphicx}

\setlength{\parindent}{0pt}
\usepackage{setspace}
\onehalfspacing

\begin{document}

\section*{IT19tb WIN7 S4 Aufgabe 3}
Leo Rudin \& Stefan Teodoropol

\subsection*{a)}

Fläche des Kreissegments: \( \frac{1}{2}r^2(\varphi - \sin\varphi) \)\\
Fläche der ungefüllten Fläche: \( \frac{1}{4}r^2\pi \)\\

Gleichung:\\
\( \frac{1}{2}r^2(\varphi - \sin\varphi) = \frac{1}{4}r^2\pi \)\\
\( \frac{1}{2}(\varphi - \sin\varphi) = \frac{1}{4}\pi \)\\
\( \varphi - \sin\varphi = \frac{1}{2}\pi \)\\
\(  \sin\varphi - \varphi = -0.5\pi \)\\

\subsection*{b)}

Gleichung umgeformt in Fixpunktiterationsgleichung:\\
\( \varphi_{n+1} = 0.5\pi + \sin(\varphi_n) \)\\\\
Laut der Skizze auf dem Aufgabenblatt sehen wir, dass der Winkel grösser sein muss als \( \frac{\pi}{2} \) und kleiner als \( \pi \) - wir wählen den Mittelwert \( \frac{3\pi}{4} \).\\\\
Wir machen die Fixpunktiteration mit \( x_0 \) = \( \frac{3\pi}{4} \):\\
\( x_1 = 0.5\pi + \sin(x_0) = 2.277903107981444 \)\\
\( x_2 = 0.5\pi + \sin(x_1) = 2.3310409238705265 \)\\
...\\
\( x_{13} = 0.5\pi + \sin(x_{12}) = 2.3096054609789665 \)\\\\
Es konvergiert also gegen: 2.3096054609789665\\
In Bogenmass:\\
\( \frac{2.28 * 360}{2\pi} \approx 132 \) Grad

\subsection*{c)}

Wir rechnen die Höhe der Ankatethe des inneren Dreiecks mit Winkel \( \frac{\varphi}{2} \) aus. Dann addieren wir den Radius zur Höhe hinzu, um die Gesamthöhe zu kriegen.\\\\
Füllhöhe: \( h(\varphi) = r + (\cos(\frac{\varphi}{2})*r) \)

\end{document}
