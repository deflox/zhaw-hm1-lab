\documentclass{article}
\usepackage[utf8]{inputenc}

\usepackage[utf8]{inputenc}
% to set spacing of page margins
\usepackage[margin=2cm,nohead]{geometry}
% to include images
\usepackage{graphicx}
% to automatically create paragraphs when doing so in the code
\usepackage{parskip}

\usepackage{amsmath}

% to not indent new paragraphs
\setlength{\parindent}{0pt}

% line spacing
\usepackage{setspace}
%\doublespacing
\onehalfspacing

\begin{document}

\section*{IT19tb WIN7 S9 Aufgabe 1}
Leo Rudin \& Stefan Teodoropol

Ax = b mit

\[
A =
\begin{bmatrix} 
1 & 0 & 2 \\
0 & 1 & 0 \\
10^{-4} & 0 & 10^{-4} \\
\end{bmatrix}
b =
\begin{bmatrix} 
1 \\
1 \\
0 \\
\end{bmatrix}
\]

Vektornorm: \(||x||_\infty = \text{max}(|x_1|,...,|x_n|)\)\\
Matrixnorm: \(||x||_\infty = \text{max}(|x_{11}|+...+|x_{1n}|,...,|x_{n1}|+...+|x_{nn}|)\)

\subsection*{a)}

\(\text{cond}(A) = ||A||_\infty * ||A^{-1}||_\infty \)

\(\text{cond}(A) = ||
\begin{bmatrix} 
1 & 0 & 2 \\
0 & 1 & 0 \\
10^{-4} & 0 & 10^{-4} \\
\end{bmatrix}
||_\infty * ||
\begin{bmatrix} 
-1 & 0 & 20'000 \\
0 & 1 & 0 \\
1 & 0 & -10'000 \\
\end{bmatrix}
||_\infty = 3 * 20'001 = 60'003\)

\subsection*{b)}

\(||b - \overset{\sim}{b}||_\infty = ||
\begin{bmatrix} 
1 \\
1 \\
0 \\
\end{bmatrix}
-
\begin{bmatrix} 
1 \\
1 \\
\varepsilon \\
\end{bmatrix}
||_\infty
=
||
\begin{bmatrix} 
0 \\
0 \\
-\varepsilon \\
\end{bmatrix}
||_\infty
=
\varepsilon
\)

\(||b||_\infty = ||
\begin{bmatrix} 
1 \\
1 \\
0 \\
\end{bmatrix}
||_\infty = 1\)

Folgende Gleichung muss erfüllt werden:

\(\frac{||x - \overset{\sim}{x}||_\infty}{||x||_\infty} \leq ||A^{-1}||_\infty * ||A||_\infty * \frac{||b - \overset{\sim}{b}||_\infty}{||b||_\infty}\)

Mit eingesetzten Werten:

\(0.01 \leq \frac{60'003\varepsilon}{1}\)\\
\(0.01 \leq 60'003\varepsilon\)\\
\(\frac{0.01}{60'003} \leq \varepsilon\)

Die Abweichung bei der z-Komponente in \(\overset{\sim}{b}\) darf höchstens \(\frac{0.01}{60'003}\) bzw. \(1.67*10^{-7}\) sein, damit die Ungleichung noch stimmt.

\newpage
\subsection*{c)}

LR-Zerlegung für A:

R:

\[
\begin{bmatrix} 
1 & 0 & 2 \\
0 & 1 & 0 \\
10^{-4} & 0 & 10^{-4} \\
\end{bmatrix}
\xrightarrow[\text{III - }10^{-4}\times\text{I}]{\text{II - }0\times\text{I}}
\begin{bmatrix} 
1 & 0 & 2 \\
0 & 1 & 0 \\
0 & 0 & -10^{-3} \\
\end{bmatrix}
\xrightarrow{\text{III - 0}\times\text{II}}
\begin{bmatrix} 
1 & 0 & 2 \\
0 & 1 & 0 \\
0 & 0 & -10^{-4} \\
\end{bmatrix}
\]

L:

\[
\begin{bmatrix} 
1 & 0 & 0 \\
0 & 1 & 0 \\
10^{-4} & 0 & 1 \\
\end{bmatrix}
\]

\textbf{Lösen für \(b\):}

Ly = b:

\(y_1 = 1\)\\
\(y_2 = 1\)\\
\(10^{-4}y_1 + y_3 = 0 \rightarrow y_3 = -10^{-4}*(1) = -10^{-4}\)

Rx = y:

\(x_1 + 2x_3 = 1 \rightarrow x_1 = 1 - 2*(1) = -1\)\\
\(x_2 = 1\)\\
\(-10^{-4}x_3 = -10^{-4} \rightarrow x_3 = 1\)

\textbf{Lösen für \(\overset{\sim}{b}\):}

Ly = b

\(y_1 = 1\)\\
\(y_2 = 1\)\\
\(10^-4y_1 + y_3 = \frac{0.01}{60'003} \rightarrow y_3 = \frac{0.01}{60'003} - 10^{-4}*(1) \)

Rx = y:

\(x_1 + 2x_3 = 1 \rightarrow x_1 = 1 - 2*(\frac{\frac{0.01}{60'003} - 10^{-4}}{-10^{-4}}) \approx -0.997\)\\
\(x_2 = 1\)\\
\(-10^{-4}x_3 = \frac{0.01}{60'003} - 10^{-4} \rightarrow x_3 = \frac{\frac{0.01}{60'003} - 10^{-4}}{-10^{-4}} \approx 0.998\)

\newpage
Abweichung:

\(||\overset{\sim}{x} - x||_\infty = ||
\begin{bmatrix} 
-0.997\\
1\\
0.998\\
\end{bmatrix}
-
\begin{bmatrix} 
-1\\
1\\
1\\
\end{bmatrix}
||_\infty
=
||
\begin{bmatrix} 
0.003\\
0\\
-0.002\\
\end{bmatrix}
||_\infty
=
0.003\)

\(||x||_\infty =
||
\begin{bmatrix} 
-1\\
1\\
1\\
\end{bmatrix}
||_\infty
= 1
\)

\(\frac{||\overset{\sim}{x} - x||_\infty}{||x||_\infty} = \frac{0.003}{1} = 0.003\)

Der tatsächliche relative Fehler beträgt 0.3\%.

\subsection*{d)}

\(\overset{\sim}{A} = 
\begin{bmatrix} 
1 + 10^{-7} & 10^{-7} & 2 + 10^{-7} \\
10^{-7} & 1 + 10^{-7} & 10^{-7} \\
10^{-4} + 10^{-7} & 10^{-7} & 10^{-4} + 10^{-7} \\
\end{bmatrix}
\)

\(||A - \overset{\sim}{A}||_\infty = || 
\begin{bmatrix} 
1 & 0 & 2 \\
0 & 1 & 0 \\
10^{-4} & 0 & 10^{-4} \\
\end{bmatrix}
-
\begin{bmatrix} 
1 + 10^{-7} & 10^{-7} & 2 + 10^{-7} \\
10^{-7} & 1 + 10^{-7} & 10^{-7} \\
10^{-4} + 10^{-7} & 10^{-7} & 10^{-4} + 10^{-7} \\
\end{bmatrix}
||_\infty\)\\
\(
 =
||
\begin{bmatrix} 
-10^{-7} & -10^{-7} & -10^{-7} \\
-10^{-7} & -10^{-7} & -10^{-7} \\
-10^{-7} & -10^{-7} & -10^{-7} \\
\end{bmatrix}
||_\infty = 3 * 10^{-7}
\)

Neue Abschätzung wenn beide Seiten gestört sind:

\(\frac{||x - \overset{\sim}{x}||_\infty}{||x||_\infty} \leq \frac{\text{cond}(A)}{1-\text{cond}(A)*\frac{||A-\overset{\sim}{A}||_\infty}{||A||_\infty}} * (\frac{||A-\overset{\sim}{A}||_\infty}{||A||_\infty} + \frac{||b - \overset{\sim}{b}||_\infty}{||b||_\infty})\)

\(0.01 \leq \frac{60'003}{1-60'000*\frac{3 * 10^{-7}}{3}} * (\frac{3 * 10^{-7}}{3} + \frac{\varepsilon}{1})\)

\(\frac{0.01}{\frac{60'003}{1-60'000*\frac{3 * 10^{-7}}{3}}} \leq \frac{3 * 10^{-7}}{3} + \varepsilon\)

\(\frac{0.01}{\frac{60'003}{1-60'000*\frac{3 * 10^{-7}}{3}}} - \frac{3 * 10^{-7}}{3} \leq \varepsilon\)

Die Abweichung bei der z-Komponente in \(\overset{\sim}{b}\) darf höchstens \(\frac{0.01}{\frac{60'003}{1-60'000*\frac{3 * 10^{-7}}{3}}} - \frac{3 * 10^{-7}}{3}\) bzw. \(\approx\) \(6.5658*10^{-8}\) sein, damit die Ungleichung noch stimmt.

\end{document}